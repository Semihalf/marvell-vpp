\documentclass{beamer}

\usepackage{amsmath}
\usepackage{amssymb}



\usecolortheme{seagull}
\titlegraphic{\includegraphics[width=\textwidth]{shirtw.png}}

\title{Introduction to VPP}
\author{Szymon Sliwa}
\date{11.05.2018}

\setbeamertemplate{section in toc}[sections numbered]

\begin{document}
\frame{\titlepage}

  \begin{frame}
  \tableofcontents
  \end{frame}

\section{What is VPP?}
  \begin{frame}
  \frametitle{What is VPP?}
  \textit{The VPP platform is an extensible framework that provides out-of-the-box production quality switch/router functionality. It is the open source version of Cisco's Vector Packet Processing (VPP) technology: a high performance, packet-processing stack that can run on commodity CPUs. \\
The benefits of this implementation of VPP are its high performance, proven technology, its modularity and flexibility, and rich feature set...} - \url{https://wiki.fd.io/view/VPP/What_is_VPP\%3F} \\
  \vfill
You can find braging about performance and summary of the project aims there

  \end{frame}

\section{Cross compilation}
  \begin{frame}
  \frametitle{Cross compilation}
  As cross compilations for Marvell platforms is quite cumbersome, a script has been 
  created to automate things. The script and a basic instructions is located at
  {\tiny\url{https://github.com/Semihalf/marvell-vpp/tree/vpp-build-tool}}
  \vfill
    When changes were made only in dpdk or musdk, it may be enough to just
  recompile dpdk and/or musdk and dpdk\_plugin, detailed instructions are included in the readme
  of the above repository.
  \end{frame}

\section{Debug capabilities}
  \begin{frame}
  \frametitle{Debug capabilites}
  If compiled natively, VPP has useful makefile targets which can be listed using
  \textbf{make} without arguments, amongst others, there are \\
  \textbf{make build} which builds debug version of VPP \\
  \textbf{make build-release} which builds release version of VPP \\
  \textbf{make run} \\
  \textbf{make run-release} \\
  \textbf{make debug} \\
  \textbf{make debug-release}

  The main difference between \textbf{debug} and \textbf{release} versions
  are the compile time flags, \textbf{release} has "-O2" and
  \textbf{debug} has "-O0 -DCLIB\_DEBUG" \tiny{("-g" is included in both)}
  \end{frame}
  
\section{Useful CLI commands}
  \begin{frame}
  \frametitle{Useful CLI commands}
  VPP has quite a lot of CLI commands, many of them useful,
  but it aids a new user in learning them:
  VPP provides \textit{tab} autocompletion and CLI help.
  To use the CLI help append \textbf{?} to a paritaly completed command
  and it will print all the possible expansions of the command with
  the needed arguments.
  
  \vfill
  
  \tiny{Also, some command abbreviations work}
  \end{frame}
      
\section{CLI commands for debugging}
  \begin{frame}
  \frametitle{CLI commands for debugging}
  \begin{itemize}
  \item \textbf{show error} shows error counters, and \textit{all} ipsec packets
  \item \textbf{trace add dpdk input} \textit{num} adds \textit{num} of packets to trace
  \item \textbf{show trace} prints the log of the packets traversing the graph 
          {\tiny very useful debugging configurational issues}
  \item \textbf{clear trace} clears trace buffers {\tiny the amount of packets to trace is also set to 0}
  \item \textbf{show version} to be sure the right version of software is tested
  \end{itemize}
  \end{frame}

\section{Running VPP}
  \begin{frame}
  \frametitle{Running VPP}
  VPP normally takes parameters from startup.conf, but on Armada platforms we
  provide it directly as command line parameter. So things like
  ports, worker amount, worker to port mapping and memory amount 
  can be changed in the script
  \textit{start\_with\_crypto\_a3k.sh}. \\
  \textbf{show int rx} show which worker polls on which port
  \end{frame}

\section{VPP Architecture}
  \begin{frame}
  \frametitle{VPP Architecture}
  "The VPP platform is built on a ‘packet processing graph’. This modular approach means that anyone can ‘plugin’ new graph nodes. This makes extensibility rather simple, and it means that plugins can be customized for specific purposes..." - \url{https://wiki.fd.io/view/VPP/What_is_VPP\%3F}
  \end{frame}
  
  \begin{frame}
  \frametitle{VPP Architecture}
  \framesubtitle{Integration with other frameworks for IO}
  \begin{itemize}
  \item VPP is able to communicate with Unix OS'es via TAP interfaces
  \item VPP is able to communicate with VM via Vhost-user interfaces
  \item There was an idea to port VPP to ODP, in a FD.io project named ODP4VPP, but it does not support
        the newest version of VPP, nor the newest version of ODP, and seems pretty much abandoned 
        {\tiny (as well as the rest of the ODP)}
  \item There appears to be a plugin working directly on the Marvell PP2 driver in src/plugins/marvell
  \end{itemize}
  \end{frame}
  
\section{Useful resources}
  \begin{frame}
  \frametitle{Useful resources}
  Generic VPP related stuff (non Marvell specific)
  \begin{itemize}
  \item \url{https://wiki.fd.io/view/VPP}
  \item \url{https://wiki.fd.io/view/Presentations}
  \item \url{https://www.youtube.com/channel/UCIJ2OP6_i1npoHM39kxvwyg} - fd.io youtube channel, the
                       videos are quite lengthy, but contain lots of information, sometimes
                       not avaliable elsewhere
  \end{itemize}
  \end{frame}

\end{document}
